\subsection{Multivariate Analysis}\label{sec:data_analysis/multivariate_analysis}

The goal of this analysis was to examine whether particular chemicals tend to appear preferentially in specific ranges of protein sequence- and surface-derived features. For each chemical, I computed the empirical distributions of these features and compared them across compounds. The underlying hypothesis is that if a chemical consistently interacts with proteins exhibiting certain structural or physicochemical characteristics, these preferences should manifest as systematic shifts in the feature distributions.

\begin{figure}[h!]
	\centering		
	\begin{subfigure}[t]{0.495\linewidth}
		\centering
		\includegraphics[width=\linewidth]{../msc-thesis-code/data/dataExploration/plots/surface_entropy_fraction_VsChem}
		\caption[Surface entropy distribution of different chemicals]{ }
		\label{fig:surface_entropy_fractionVsChem}
	\end{subfigure}
	\begin{subfigure}[t]{0.495\linewidth}
		\centering
		\includegraphics[width=\linewidth]{../msc-thesis-code/data/dataExploration/plots/2d_hist_plot}
		\caption[Kyte-Doolittle hydropathy vs. Aromaticity for salt, buffer and \gls{peg}]{ }
		\label{fig:2d_hist_plot}
	\end{subfigure}
	\caption[Distribution analysis]{The figure demonstrates that surface (\autoref{fig:surface_entropy_fractionVsChem}) and sequence feature distributions encode predictive information regarding the presence of particular chemicals. Although the observed differences are subtle, distinct shifts in these distributions are associated with variations in the probability of specific chemicals occurring. Similar to how \autoref{fig:2d_hist_plot} shows the 2D distributions for different chemicals, the model would combine multiple features to differentiate between the chemicals.}
	
	\label{fig:distribution_analysis}
\end{figure}

Across most surface and sequence features, the observed differences between chemicals were present but relatively subtle. For many features, the distributions of different chemicals largely overlapped and diverged only in localized regions (\autoref{fig:surface_entropy_fractionVsChem}). This indicates that no single feature alone is strongly predictive. However, when examining two features jointly, the separability increases. \\
The bivariate distributions (\autoref{fig:2d_hist_plot}) already show clearer clustering for certain compound classes, suggesting that the underlying relationships between chemicals and protein properties are inherently multivariate. In other words, although each single feature contains only weak signal, the combination of multiple features provides stronger discriminatory power.


\begin{figure}[h!]
	\centering		
	\begin{subfigure}[t]{0.495\linewidth}
		\centering
		\includegraphics[width=\linewidth]{../msc-thesis-code/data/dataExploration/plots/surfaceCompoundDistributions_sasa_skewness_hydrophobic_surface_fraction_surface_entropy_fraction_net_charge_normalized}
		\caption[Difference in average sequence feature for chemicals]{ }
		\label{fig:surfaceCompoundDistributions_sasa_skewness_hydrophobic_surface_fraction_surface_entropy_fraction_net_charge_normalized}
	\end{subfigure}
	\begin{subfigure}[t]{0.495\linewidth}
		\centering
		\includegraphics[width=\linewidth]{../msc-thesis-code/data/dataExploration/plots/sequenceChemDistributions_pI_mw_arom_gravy}
		\caption[Difference in average surface feature for chemicals]{ }
		\label{fig:sequenceChemDistributions_pI_mw_arom_gravy}
	\end{subfigure}
	\caption[Distribution analysis]{The point plot displays the z-score normalized median values of each feature, with standard errors, across different chemicals. They are derivative of distributions as seen in \autoref{fig:surface_entropy_fractionVsChem}. Features with larger or more distinct deviations reflect stronger predictive influence on the occurrence of specific chemicals. Deviations for certain chemicals occur both in surface features (\ref{fig:surfaceCompoundDistributions_sasa_skewness_hydrophobic_surface_fraction_surface_entropy_fraction_net_charge_normalized}) as well as (\ref{fig:sequenceChemDistributions_pI_mw_arom_gravy}.)}
	
	\label{fig:medianAnalysisSequenceAndSurface}
\end{figure}


From these univariate distributions, I next computed the median and its standard error for each feature–chemical pair and normalized them to z-scores. These aggregated statistics visualize how the typical feature value shifts depending on the chemical (\autoref{fig:medianAnalysisSequenceAndSurface}). Some features exhibit noticeably higher between-chemical variability than others, implying that they carry more predictive information. For example, the sequence-derived molecular weight differs substantially across chemicals, whereas other features remain close to the overall mean. Similarly, certain chemicals deviate strongly from the global average regardless of the feature category. \gls{peg} 400, in particular, shows large positive shifts for both the hydrophobic surface fraction and the Kyte–Doolittle hydropathy index.

This agreement is expected because the Kyte–Doolittle hydropathy is derived from the intrinsic hydrophobicity of amino acids in the primary sequence, whereas the hydrophobic surface fraction quantifies how much of the protein’s solvent-exposed area is contributed by hydrophobic residues. Proteins that contain hydrophobic, high-GRAVY segments tend to bury some hydrophobic residues but still expose proportionally more non-polar surface patches than proteins with overall polar sequences. Thus, these two features capture related biochemical tendencies even though they were derived from different structural levels.


\begin{figure}[h!]
	\centering		
	\begin{subfigure}[t]{0.495\linewidth}
		\centering
		\includegraphics[width=\linewidth]{../msc-thesis-code/data/dataExploration/plots/ks_distance_all_sequence}
		\caption[Sequence feature ks distance of chemicals]{Differences of distributions for all sequence features between chemicals.}
		\label{fig:ks_distance_all_sequence}
	\end{subfigure}
	\begin{subfigure}[t]{0.495\linewidth}
		\centering
		\includegraphics[width=\linewidth]{../msc-thesis-code/data/dataExploration/plots/ks_distance_all}
		\caption[Surface feature ks distance of chemicals]{Differences of distributions for all surface features between chemicals.}
		\label{fig:ks_distance_all_surface}
	\end{subfigure}
	\caption[Distribution difference]{The KS distance between the distributions of one chemical to every other chemical for the 15 most common compounds. The KS distance was calculated for every feature independently. All distances were then averaged. }
	
	\label{fig:ks_distance_all_features}
\end{figure}

Finally, to quantify how different the full distributions are across chemicals, I computed the pairwise \gls{ks} distances for all surface and sequence features and averaged them (\autoref{fig:ks_distance_all_features}). This analysis highlights which chemicals are globally most distinguishable from the others. \gls{peg} 400 consistently exhibits the highest \gls{ks} distances across nearly all surface features, confirming that its usage correlates with a characteristic and unusual pattern of protein properties. Sodium acetate also shows elevated distances, suggesting it may likewise be more predictable than most other compounds.

The \gls{ks} analysis further reveals chemical pairs that are either highly similar or strongly divergent. For instance, Tris and Tris-HCl are extremely similar, which is expected because they are chemically almost identical buffering agents; the only difference is protonation state, and both are typically used in similar pH ranges. In contrast, Tris differs sharply from MES or HEPES, both of which are sulfonic-acid–based buffering molecules with distinct pKa values and chemical behavior. These compounds stabilize proteins under different physicochemical conditions, which likely explains the characteristic shifts observed in their associated protein feature distributions.

Together, these results indicate that the problem of predicting chemicals from protein properties is not dominated by any single strongly discriminative feature but is instead inherently multivariate. Predictive information arises from subtle but consistent patterns across several sequence- and structure-derived features, supporting the use of machine-learning methods capable of jointly modeling these interactions.




