\section{Data Analysis}\label{sec:data_analysis}

\subsection{Univariate Analysis}\label{sec:data_analysis/univariate_analysis}
Histograms, distributions, summary statistics

\subsubsection{Input}
\begin{figure}[h!]
	\centering
	
	\begin{subfigure}[t]{0.32\linewidth}
		\centering
		\includegraphics[width=\linewidth]{../msc-thesis-code/data/dataExploration/plots/sequenceLength}
		\caption[Sequence Length Distribution]{Sequence Length Distribution}
		\label{fig:seqLength}
	\end{subfigure}
	\begin{subfigure}[t]{0.32\linewidth}
		\centering
		\includegraphics[width=\linewidth]{../msc-thesis-code/data/dataExploration/plots/atomCount}
		\caption[Atom count Distribution]{Atom count Distribution}
		\label{fig:atomCounts}
	\end{subfigure}
	\begin{subfigure}[t]{0.32\linewidth}
		\centering
		\includegraphics[width=\linewidth]{../msc-thesis-code/data/dataExploration/plots/atomCountVsSequenceLength}
		\caption[Atom count vs. Sequence Length]{Atom count vs. Sequence Length}
		\label{fig:atomCountsVsSeqLength}
	\end{subfigure}
	\caption[Sequence Length and Atom count Distribution]{Visualization of the distribution of sequence lengths (\autoref{fig:seqLength}) and deposited atom counts (\autoref{fig:atomCounts}). Most proteins have sequences between 50 and 400 residues, with models containing roughly 1,000–8,000 atoms. The relationship is largely linear as seen in \autoref{fig:atomCountsVsSeqLength}, though missing or extra atoms in some structures cause deviations in slope.}
	\label{fig:sequenceAndAtomCount}
\end{figure}


\begin{figure}[h!]
	\centering		
	\begin{subfigure}[t]{0.49\linewidth}
		\centering
		\includegraphics[width=\linewidth]{../msc-thesis-code/data/dataExploration/plots/aminoAcidOccurence}
		\caption[Residue distribution]{Distribution of amino acids in the sequences. Residue frequencies vary substantially—Leucine is over four times more common than Tryptophan.}
		\label{fig:aminoAcidOccurence}
	\end{subfigure}
	\hspace{0.1cm}
	\begin{subfigure}[t]{0.49\linewidth}
		\centering
		\includegraphics[width=\linewidth]{../msc-thesis-code/data/dataExploration/plots/observedVsPredictedAminoAcidCount}
		\caption[Observed vs. Predicted Occurrences of Residues]{Observed vs.\ predicted residue frequencies, illustrating the relationship between amino-acid occurrence and the codon patterns that encode them.}
		\label{fig:observedVsPredictedAminoAcidCount}
	\end{subfigure}
	\caption[Residue analysis in protein sequences]{
		\autoref{fig:aminoAcidOccurence} shows the distribution of amino acids across all sequences, while \autoref{fig:observedVsPredictedAminoAcidCount} compares observed residue frequencies with those predicted from codon usage.}
	
	\label{fig:aminoAcidCounts}
\end{figure}


\begin{figure}[h!]
	\centering		
	\begin{subfigure}[b]{0.495\linewidth}
		\centering
		\includegraphics[width=\linewidth]{../msc-thesis-code/data/dataExploration/plots/gravyHist}
		\caption[Kyte-Doolittle hydropathy distribution]{Kyte-Doolittle hydropathy distribution}
		\label{fig:gravyHist}
	\end{subfigure}
	\begin{subfigure}[b]{0.495\linewidth}
		\centering
		\includegraphics[width=\linewidth]{../msc-thesis-code/data/dataExploration/plots/aromHist}
		\caption[Aromaticity distribution]{Aromaticity distribution}
		\label{fig:aromaticity}
	\end{subfigure}
	\begin{subfigure}[b]{0.495\linewidth}
		\centering
		\includegraphics[width=\linewidth]{../msc-thesis-code/data/dataExploration/plots/pIHist}
		\caption[Isoelectric point distribution]{Isoelectric point distribution}
		\label{fig:pIHist}
	\end{subfigure}
	\begin{subfigure}[b]{0.495\linewidth}
		\centering
		\includegraphics[width=\linewidth]{../msc-thesis-code/data/dataExploration/plots/mwHist}
		\caption[Molecular weight distribution]{Molecular weight distribution}
		\label{fig:mwHist}
	\end{subfigure}
	\caption[Distributions of sequence-derived features]{Distributions of sequence-derived features across the dataset. 
	Kyte--Doolittle hydropathy scores are roughly normally distributed with a slightly negative mean, while aromaticity values form a mildly positive normal distribution. 
	The isoelectric point shows a bimodal pattern with peaks near pH~6 and pH~9. 
	Molecular weights decrease sharply beyond 50{,}000~Da, mirroring trends in sequence length and atom count seen in \autoref{fig:sequenceAndAtomCount}.}

	\label{fig:sequenceInformation}
\end{figure}

\begin{figure}[h!]
	\centering		
	\begin{subfigure}[t]{0.495\linewidth}
		\centering
		\includegraphics[width=\linewidth]{../msc-thesis-code/data/dataExploration/plots/distribution_apolar_vs_polar}
		\caption[Apolar and Polar Surface Atoms]{Distribution of apolar vs. polar surface atoms.}
		\label{fig:apolarVsPolar}
	\end{subfigure}
	\begin{subfigure}[t]{0.495\linewidth}
		\centering
		\includegraphics[width=\linewidth]{../msc-thesis-code/data/dataExploration/plots/distribution_surface_information}
		\caption[Surface feature comparison]{Comparison of different surface features.}
		\label{fig:surfaceDistribution}
	\end{subfigure}
	\caption[Derived surface feature distributions]{Distributions of derived surface features. 
		Apolar and polar surface atoms show approximately normal distributions with comparable overall exposure. 
		Surface roughness, sidechain fraction, and SASA skewness are centered around~1, while the remaining surface descriptors cluster near zero or show slight positive shifts.}
	
	\label{fig:derivedSurfaceFeatures}
\end{figure}


\subsubsection{Label}
After parsing it can be seen that the descriptions that do not contain any information about the chemical cocktail is \roughly 3.5 \%. They are very similar to entry 3PCA in that they only define the pH or temperature. Additionally, concentrations were attributed to chemicals in \roughly 88\% of the times. 

\begin{figure}[h!]
	\centering		
	\begin{subfigure}[b]{0.495\linewidth}
		\centering
		\includegraphics[width=\linewidth]{../msc-thesis-code/data/dataExploration/plots/temperatureDistribution}
		\caption[Temperature Distribution]{Temperature Distribution}
		\label{fig:temperatureDistrbution}
	\end{subfigure}
	\begin{subfigure}[b]{0.495\linewidth}
		\centering
		\includegraphics[width=\linewidth]{../msc-thesis-code/data/dataExploration/plots/phDistribution}
		\caption{pH Distribution}{PH Distribution}
		\label{fig:pHDistribution}
	\end{subfigure}
	\caption[Temperature and pH distributions]{Distribution of numerical fields in the condition category. Only a few distinct temperatures are reported, with 35\% of samples specifying room temperature and most remaining values clustered around it; a secondary peak appears at 4\,$^\circ$C. In contrast, pH values follow an approximately normal distribution centered near pH\roughly7.}
	\label{fig:temperatureAndPh}
\end{figure}

\begin{figure}[h!]
	\centering		
	\begin{subfigure}[b]{0.42\linewidth}
		\centering
		\includegraphics[width=\linewidth]{../msc-thesis-code/data/dataExploration/plots/nCompoundsDistribution}
		\caption[Number of compounds in a cocktail]{Number of compounds in a cocktail is on average 3 but sometimes specifies as much as 10 chemicals.}
		\label{fig:nCompoundsDistribution}
	\end{subfigure}
	\hspace{0.1cm}
	\begin{subfigure}[b]{0.55\linewidth}
		\centering
		\includegraphics[width=\linewidth]{../msc-thesis-code/data/dataExploration/plots/mostCommonchemicalCompounds_withConcentrationGiven}
		\label{fig:mostCommonChemicals}
		\caption[Most common compounds with concentration]{The most common compounds across all cocktails feature multiple PEGs, buffers and salts. For the majority of the compounds a concentration value is given.}
	\end{subfigure}
	\label{fig:cocktailSizeAndMostCommonCompounds}
	\caption[General Cocktail composition]{General Cocktail composition}
\end{figure}


\begin{figure}[h!]
	\centering		
	\begin{subfigure}[t]{0.48\linewidth}
		\centering
		\includegraphics[width=\linewidth]{../msc-thesis-code/data/dataExploration/plots/most_common_pegs_with_concentration_point}
		\caption[Most common PEGs]{Most common PEGs and the average concentration value at which they occur}
		\label{fig:most_common_pegs_with_concentration_point}
	\end{subfigure}
		\hspace{0.1cm}
	\begin{subfigure}[t]{0.48\linewidth}
		\centering
		\includegraphics[width=\linewidth]{../msc-thesis-code/data/dataExploration/plots/most_common_chem_with_concentration_point}
		\caption[Most common chemicals]{Most common chemicals and the average concentration value at which they occur}
		\label{fig:most_common_chem_with_concentration_point}
	\end{subfigure}
	\caption[Most common compounds with concentrations]{General Cocktail composition}
	\label{fig:mostCommonCompoundsAndConcentration}
\end{figure}

\begin{figure}[h!]
	\centering		
	\begin{subfigure}[t]{0.48\linewidth}
		\centering
		\includegraphics[width=\linewidth]{../msc-thesis-code/data/dataExploration/plots/commonChemCombinations}
		\caption[Most common cocktails]{Most frequent specific chemical combinations found in the crystallization conditions, showing the dominant pairs and triplets of reagents across the dataset. }
		\label{fig:mostcommoncocktails}
	\end{subfigure}
	\hspace{0.1cm}
	\begin{subfigure}[t]{0.48\linewidth}
		\centering
		\includegraphics[width=\linewidth]{../msc-thesis-code/data/dataExploration/plots/commonChemClasses}
		\caption[Cocktail classes]{Most frequent combinations of chemical classes occurring together in crystallization setups, highlighting the typical pairing patterns between polymers, buffers, and salts. }
		\label{fig:commonChemClasses}
	\end{subfigure}
	\label{fig:cocktailAnalysis}
\end{figure}

\begin{figure}[h]
	\centering
	\includegraphics[width=1\linewidth]{../msc-thesis-code/data/dataExploration/plots/mostCommonPartners}
	\caption[Most common 5-compound subsets]{Most common 5-compound subsets within chemical sets. 
	For several highly frequent chemicals—including PEG~3350, ammonium sulfate, PEG~4000, HEPES, and sodium chloride—the plot shows the compounds most often co-occurring with them in crystallization conditions, highlighting typical pairing patterns among buffers, salts, polymers, and additives.}

	\label{fig:mostcommonpartners}
\end{figure}


\subsection{Multivariate Analysis}\label{sec:data_analysis/multivariate_analysis}
Feature-feature relationships, correlation matrices, scatter plots

\subsection{Outlier and Anomaly Detection}\label{sec:data_analysis/outlier_and_anomaly_detection}
Identifying unusual values or patterns



