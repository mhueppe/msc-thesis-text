
\section{CRIMS}\label{sec:crims}
In addition to the Protein Data Bank–derived dataset described in \autoref{sec:protein_database}, I incorporated a second, complementary data source: the \gls{crims} used at the \gls{embl}. \gls{crims} is a laboratory information system designed to record, track, and analyze crystallization experiments performed on-site, particularly those carried out at synchrotron beamlines. Unlike large public repositories, \gls{crims} captures local experimental outcomes for a defined set of proteins studied within a specific laboratory environment. However, the data that is recorded at \gls{embl} and other crystallization beamline providers using \gls{crims} is not public. Consequently, the number of proteins represented is smaller as it only covers proteins crystallized from the Itzen work group. However, the data quality and experimental granularity are substantially higher.

A key advantage of \gls{crims} is that it provides both positive and negative crystallization outcomes, which the \gls{pdb} inherently lacks. While the \gls{pdb} contains only successful crystallization conditions—i.e., those that yielded a structure—\gls{crims} logs the full experimental matrix. Crystallization trials are typically conducted in four 96-condition plates, and \gls{crims} records for each condition whether crystals were observed or not. Additionally, the conditions are not free text but rather in a uniform categorical/numerical format. This transforms the dataset into a true gold standard for supervised learning, as it provides clean explicit negative samples for model evaluation.

The crystallization drops are systematically monitored over time. Automated imaging systems photograph each drop at multiple intervals—after one hour, after one to three days, after one to four weeks, and up to approximately four months. These time-resolved images enable the use of machine-learning models to classify crystallization outcomes based on visual evidence of crystal formation. Because each condition is linked to the sequence and structural properties of the protein, the system allows us to determine how well a model can discriminate between crystallization conditions that lead to crystal growth and those that do not.

For the purposes of this thesis, \gls{crims} therefore serves as a critical validation dataset. It allows us to test whether a model trained on sequence- and surface-derived features can generalize beyond predicting which chemicals tend to appear in successful \gls{pdb} crystallizations and instead evaluate the actual effectiveness of crystallization conditions for specific proteins. As a result, \gls{crims} provides a robust, experimentally grounded benchmark for assessing the model's ability to distinguish productive from non-productive crystallization conditions.


\begin{figure}[t]
	\centering
	% Row 1
	\begin{subfigure}[b]{0.24\textwidth}
		\includegraphics[width=\textwidth]{media/FORMULATRIX_SD30010705_2_01-10-2025_03_03_01_00_99_Vis-1}
	\end{subfigure} \hspace{0.01cm}% <---- reduce horizontal space
	\begin{subfigure}[b]{0.24\textwidth}
		\includegraphics[width=\textwidth]{media/FORMULATRIX_SD30010708_4_07-10-2025_10_06_01_00_99_Vis}
	\end{subfigure}
	\begin{subfigure}[b]{0.24\textwidth}
		\includegraphics[width=\textwidth]{media/FORMULATRIX_SD30010708_4_07-10-2025_12_08_01_00_99_Vis}
	\end{subfigure}
	\begin{subfigure}[b]{0.24\textwidth}
		\includegraphics[width=\textwidth]{media/FORMULATRIX_SD30010708_4_07-10-2025_06_04_01_00_99_Vis}
	\end{subfigure}				
	\caption[\gls{crims} examples]{Scores: 0.99, 0.97, 0.90, 0.3}
	\label{fig:crystallization_drops_example}
\end{figure}


Unfortunately, \gls{crims} does not provide a bulk download option for crystallization experiment data. For this reason, a custom web scraper and parser were implemented. Because \gls{crims} is not a static website its content dynamically changes based on user interactions. The scraper had to simulate a real user session and systematically “check out” each crystallization plate. For every condition on each plate, both the chemical formulation and all available time-resolved images were extracted.

These images were required to obtain an objective quality assessment of each crystallization condition. To this end, a convolutional neural network developed by \textcite{King2024} was applied to score the probability that a crystal is present in a given drop. \autoref{fig:crystallization_drops_example} shows representative examples of these crystallization drops alongside their predicted scores. The model exhibits high sensitivity, meaning false negatives are extremely rare. This behavior is desirable in crystallization screening, where failing to detect a true crystal is far more problematic than incorrectly flagging a non-crystal image as positive.
