\subsection{Condition Embedding}\label{methods/feature_engineering/condition_embedding}

To embed crystallization conditions in a semantically meaningful way, this thesis employs the E5 sentence transformer architecture, a family of text-embedding models designed for high-quality retrieval and semantic similarity tasks. E5 models follow the dual-encoder paradigm introduced in the original work by \textcite{Wang2022}. In this framework, text is processed by a Transformer encoder that outputs a dense vector representation. These vectors are optimized so that semantically related texts lie close together in embedding space, while unrelated texts are pushed apart.


The starting was the pre-trained e5-small-v2 model. During fine-tuning, E5 uses contrastive learning, where pairs of texts are treated as positive or negative examples. The model is trained with an InfoNCE-style loss, encouraging the embeddings of matching or semantically equivalent texts to become similar. In the original E5 formulation, these positives come from paired query–document texts. In this thesis, the same principle is adapted to crystallization conditions.

For each free-text crystallization condition, the corresponding “label” is a parsed canonical representation containing only structured components such as compound names, concentrations, pH, and temperature obtained by the parsing paradigm described in \autoref{sec:data_parsing}. The free-text description and its canonical counterpart form a positive pair, while all other condition–label pairs act as negatives. This training setup directs the model to learn the underlying chemical relations between textual descriptions and their structured physical parameters.

\begin{table}[h]
	\centering
	\renewcommand{\arraystretch}{0.8} % reduce row height
	\begin{tabular}{c|l|c}
		\textbf{ID} & \textbf{Components} & \textbf{pH} \\ \hline
		1  & sodium malonate, PEG 3350 & 7.0 \\
		2  & potassium thiocyanate, PEG MME 2000 & None \\
		3  & -- & 7.0 \\
		4  & magnesium acetate, sodium cacodylate, PEG 8000 & 6.5 \\
		5  & sodium acetate, sodium formate & 4.6 \\
		6  & potassium sodium tartrate, PEG 3350 & None \\
		7  & ammonium citrate tribasic, PEG 3350 & 7.0 \\
		8  & sodium formate, PEG 3350 & None \\
		9  & MES, PEG 20000 & 6.5 \\
		10 & ammonium sulfate, bis-tris, PEG 3350 & 5.5 \\
		11 & sodium acetate, sodium cacodylate, PEG 8000 & 6.5 \\
		12 & bis-tris, PEG 3350 & 6.5 \\
		13 & magnesium formate, PEG 3350 & None \\
		14 & magnesium sulfate & None \\
		15 & bis-tris, PEG 3350 & 5.5 \\
	\end{tabular}
	\caption[Condition Examples]{Different conditions parsed from the free text in the \gls{pdb}}
	\label{table:conditionExamples}
\end{table}


To evaluate how well the model captures the underlying similarity between crystallization conditions, I compared 15 condition sets against each other using both a t-SNE visualization and a cosine-similarity heatmap. For each condition, I selected 20 free-text descriptions that were maximally different from one another while still mapping to the same underlying condition set. The conditions are specified in \autoref{tab:conditionExamples}. 
\begin{figure}[h!]
	\centering		
	\begin{subfigure}[t]{0.495\linewidth}
		\centering
		\includegraphics[width=\linewidth]{../msc-thesis-code/data/dataExploration/plots/conditionEmbedding_new}
		\caption[Condition Embedding]{ }
		\label{fig:conditionEmbedding}
	\end{subfigure}
	\begin{subfigure}[t]{0.495\linewidth}
		\centering
		\includegraphics[width=\linewidth]{../msc-thesis-code/data/dataExploration/plots/cosineSimilarityHeatmapConditionSet}
		\caption[Cosine Similarity Analysis]{ }
		\label{fig:cosineSimilarityHeatmapConditionSet}
	\end{subfigure}
	\caption[Condition Embedding Analysis]{\autoref{fig:conditionEmbedding} shows the t-SNE–reduced embeddings of the condition sets listed in \autoref{table:conditionExamples}.
		\autoref{fig:cosineSimilarityHeatmapConditionSet} displays the pairwise cosine similarities between condition sets.
		For each condition, 20 free-text samples were generated that all mapped to the same underlying condition set.
		As shown, the resulting embeddings form well-separated clusters in which samples belonging to the same condition set exhibit the highest mutual similarity.}	
	\label{fig:conditionEmbeddingAnalysis}
\end{figure}

As shown in \autoref{fig:conditionEmbedding}, these free-text variants form tight clusters in the t-SNE–reduced embedding space, demonstrating that the model consistently recognizes the same condition despite substantial variation in wording. This is further reflected in the cosine similarity matrix in \autoref{fig:cosineSimilarityHeatmapConditionSet}, where samples belonging to the same condition set exhibit the highest mutual similarity.

Importantly, the embedding space also preserves meaningful chemical relationships between different conditions. Condition pairs that differ only by closely related compounds cluster in close proximity—for example, conditions 8 and 13 (sodium formate vs. magnesium formate) and conditions 4 and 11 (magnesium acetate vs. sodium acetate). This indicates that the model captures not only textual similarity but also chemical interchangeability among related components. Overall, the results suggest that the embedding is largely agnostic to the specific way a condition is expressed in free text while still reflecting chemically relevant similarities between distinct condition sets.

As a result, the model produces embeddings that capture the essential chemical and experimental features of crystallization conditions. Free-text descriptions that encode similar compounds, concentration ranges, or physicochemical properties are embedded close to one another, enabling meaningful comparison, clustering, and downstream prediction. 