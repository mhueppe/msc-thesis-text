\subsection{Condition Embedding}\label{methods/feature_engineering/condition_embedding}

To embed crystallization conditions in a semantically meaningful way, this thesis employs the E5 sentence transformer architecture, a family of text-embedding models designed for high-quality retrieval and semantic similarity tasks. E5 models follow the dual-encoder paradigm introduced in the original work by \textcite{Wang2022}. In this framework, text is processed by a Transformer encoder that outputs a dense vector representation. These vectors are optimized so that semantically related texts lie close together in embedding space, while unrelated texts are pushed apart.


The starting was the pre-trained e5-small-v2 model. During fine-tuning, E5 uses contrastive learning, where pairs of texts are treated as positive or negative examples. The model is trained with an InfoNCE-style loss, encouraging the embeddings of matching or semantically equivalent texts to become similar. In the original E5 formulation, these positives come from paired query–document texts. In this thesis, the same principle is adapted to crystallization conditions.

For each free-text crystallization condition, the corresponding “label” is a parsed canonical representation containing only structured components such as compound names, concentrations, pH, and temperature obtained by the parsing paradigm described in \autoref{sec:data_parsing}. The free-text description and its canonical counterpart form a positive pair, while all other condition–label pairs act as negatives. This training setup directs the model to learn the underlying chemical relations between textual descriptions and their structured physical parameters.


\begin{figure}[h]
	\centering
	\includegraphics[width=0.9\linewidth]{../msc-thesis-code/data/dataExploration/plots/conditionEmbedding}
	\caption[Condition Embedding]{Embedding of the language model for the condition sets.}
	\label{fig:conditionembedding}
\end{figure}

As a result, the model produces embeddings that capture the essential chemical and experimental features of crystallization conditions. Free-text descriptions that encode similar compounds, concentration ranges, or physicochemical properties are embedded close to one another, enabling meaningful comparison, clustering, and downstream prediction. 
\autoref{fig:conditionembedding} shows some example conditions and their dimension-reduced embeddings. 